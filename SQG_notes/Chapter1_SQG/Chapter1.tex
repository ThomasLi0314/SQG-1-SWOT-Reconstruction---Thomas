\documentclass[../SQG_notes.tex]{subfiles}

\begin{document}
    \section{Quasi-Geostrophic Equations}
    In this section, I introduce some basic equations in QG theories. Building a foundation for SQG, eSQG and other variations introduced later. The analysis is already in a \textbf{stratified ocean}. 
    \subsection{Governing Equations}
    Momentum Equation
    \begin{equation}
        \frac{\D \bu}{\D t} + \mathbf{f} \times \bu = -\frac{\nabla p}{\rho} \label{momentum equation}
    \end{equation}
    Mass Conservation 
    \begin{equation}
        \frac{\D \rho}{\D t} + \rho \nabla \cdot \mathbf{v} = 0
    \end{equation}
    We are in a \textbf{stratified ocean}. Breaking the total state variables into a "hydrostatic reference state" (which depends only on $z$) and a "dynamic perturbation" (which moves the fluid): 
    
    \begin{equation}
        \rho = \tdrho(z) + \rho_1(x,y,z,t) \label{stratified density}
    \end{equation}
    and 
    \begin{equation}
        p = p_0(z) + p_1(x,y,z,t) \label{stratifies pressure}
    \end{equation}
    Then RHS of Eq \ref{momentum equation} becomes 
    \[ -\frac{1}{\rho} \nabla p_1 \sim -\frac{1}{\rho_0}\nabla p_1\]
    Define the \textbf{Kinematic Pressure}
    \begin{equation}
        \phi = \frac{p_1}{\rho_0}
    \end{equation}
    Momentum Equation becomes 
    \begin{equation}
        \boxed{\frac{\mathrm{D} \bu}{\mathrm{D} t}+\mathbf{f} \times \bu = -\nabla \phi} \label{momentum equation}
    \end{equation}
    Hydrostatic balance is a state of equilibrium in a fluid where the upward force of pressure exactly balances the downward force of gravity.
    \begin{equation}
        -g\tdrho = \frac{\dr p_0}{\dr z} \label{hydrostatic balance}
    \end{equation}
    \subsubsection{Continuity Equation Approximatio}
    The genearl continuity equation can always be expressed as 
    \[ \frac{\D \rho}{\D t} + \rho \nabla \cdot u = 0\]
    In Ocean, we assume that the fluid is \textbf{Incompressible} and wrote
    \[ \frac{\D \rho}{\D t} = 0 \qquad \nabla \cdot \bu = 0\]
    In Atmosphere, we use the \textbf{Anelastic Assumption}
    Then the mass conservation yeilds
    \[ \frac{\p \rho_1}{\p t} + \nabla \cdot ((\tdrho + \rho_1)\mathbf{v}) = 0 \qrq \boxed{\nabla \cdot (\tdrho\mathbf{v}) = 0} \]
    \begin{remark}
        Here we drop the $\p_t \rho_1$ term under the \textbf{Anelastic assumption}. Essentially by eliminating this partial derivative, we assume the fluid is anelastic, so sound wave is not supported \mn{or less important} in the system. However, this approximation is normally used for \textbf{deep atmospheric or stellar convertion} where density $\tilde{\rho}$ changes significantly with height.
    \end{remark}
    For Oceanography, 
    Next we define the Buoyancy :
    \begin{equation}
        b = -g\frac{\rho}{\bar{\rho}_0} \label{Buoyancy}
    \end{equation}
    Here $\brho$ is a constant reference density. The the divergent free condition implies 
    \begin{equation}
        \boxed{\frac{\D b}{\D t} = 0} \label{Buoyancy equation}
    \end{equation}
    And thus the Hydrostatic balance equation Eq \ref{hydrostatic balance} implies 
    \begin{equation}
        \boxed{\frac{\p \phi}{\p z} = b}
    \end{equation}
    All the Boxed Equation together is the \textbf{Hydrostatic Anelastic Equations for Stratified Flow}. If we consider the perturbation of Buoyancy 
    \[ b = \tilde{b}(z) + b_1(x,y,z,t)\]
    Expand Eq \ref{Buoyancy equation} can be writen as 
    \begin{equation}
        \boxed{\frac{\D b_1}{\D t} + w \frac{\dr b}{\dr z} = 0}  \mn{this is the more familiar buoyancy equation we see in lecture}
    \end{equation}
    In a more familiar form we define 
    \[ N^2 = \frac{\dr b}{\dr z} = -g \frac{\tdrho_z}{\brho}\]
    Which is the \textbf{Brunt Vasala Frequency}.

    \subsection{Scaling Analysis}
    To simplify our equation, we introduce some scalings. 
    \[
    (x, y) \sim L, \quad(u, v) \sim U, \quad t \sim \frac{L}{U}, \quad z \sim H, \quad f \sim f_0
    \]
    Introduce the \textbf{Rosbby Number}:
    \begin{equation}
        \text{Ro} = \frac{U}{f_0 L}
    \end{equation}
    Now let $\phi = \tilde{\phi}(z) + \phi_1(x,y,z,t)$. Then since the gradient in Eq \ref{momentum equation} is horizontal, we can replace $\phi$ by $\phi_1$. Now suppose  
    \[
    |\mathbf{f} \times \bu| \sim\left|\nabla \phi_1\right|
    \]
    From Hydrostatic balance we have 
    \[ b \sim \frac{f_0 UL}{H}\]
    Then 
    \[
    \frac{\left(\partial b^{\prime} / \partial z\right)}{N^2}\sim \frac{f_0 UL}{(HN)^2} \sim R o \frac{L^2}{L_d^2}
    \]
    Where we have the deformation radius as a function of $z$. 
    \[ L_d = \frac{NH}{f_0}\]
    Introduce dimensionless variables
    \[
(\widehat{u}, \widehat{v})=U^{-1}(u, v) \quad \widehat{w}=\frac{L}{U H} w, \quad \widehat{f}=f_0^{-1} f, \quad \widehat{\phi}=\frac{\phi_1}{f_0 U L}, \quad \widehat{b}=\frac{H}{f_0 U L} b_1
    \]
    \begin{remark}
    We then have dimensionless equation of motion for Atmosphere or Ocean.  \mn{often people drop the hat for simplicity. However in the first derivation I keep everything with a hat.}
        \begin{align}
            \text{Momentum Equation} :\quad & \Ro \frac{\D \widehat{\bu}}{\D t} + \hatf \times \hatu = -\nabla \hatphi \label{dimensionless momentum equation} \\
            \text{Buoyancy Equation} :\quad& \Ro \frac{\D \widehat{b}}{\D t} + \Bu \widehat{w} = 0 \label{dimensionless buoyancy equation}\\
            \text{Hydrostatic Balance}:\quad & \frac{\p \hatphi}{\p \widehat{z}} = \widehat{b} \label{dimensionless hydrostatic balance equation}\\
            \text{Continuity(\textbf{Atmosphere})} :\quad& \widehat{\nabla} \cdot \hatu + \frac{1}{\tdrho} \frac{\p \tdrho \widehat{w}}{\p \widehat{z}} = 0 \label{dimensionless continuity equation for atmosphere}\\ 
            \text{Continuity(\textbf{Oceanography})} : \quad& \widehat{\nabla} \cdot \widehat{\bu} = 0 \label{dimensionless contiuity equation for ocean}
        \end{align}
        From now on I will drop the hats.
    \end{remark}
    \subsection{Quasi-Geostrophic Potential Vorticity Equation}
    We now derive the Quasi-Geostrophic Potential Vorticity Equations. Starting from asymptotic expansions \mn{hat is dropped}
    \[ \bu = (u, v, w) = \bu_g + \Ro \bu_1 \quad \phi = \phi_0 + \Ro \phi_1 \quad b = b_0 + \Ro b_1\]
    Here we consider the $\beta$ effect. 
    \[ \mathbf{f} = f_0 \bm{k} + \beta y \bm{k}\]
    Let $\epsilon = \Ro$.
    \textbf{Momentum Equation} : \par
    The $O(1)$ momentum equation gives the Geostrophic balance
    \begin{equation}
        f_0 \bm{k} \times \bu_g = -\nabla \phi_0 \label{Geostrophic balance equation}
    \end{equation}
    
    Immediately this implies 
    \[ \nabla \cdot \bu_g = 0\]
    And $O(\epsilon)$ is 
    \begin{equation}
        \frac{\D_g \bu_g}{\D t} + \beta y \bm{k} \times \bu_g  + f_0\bm{k} \times \bu_1 = - \nabla \phi_1 \label{first order momentum equation}
    \end{equation}
    Here $\D_g$ is the geostrophic material derivative
    \[ \D_g = \p_t + \bu_g \cdot \nabla \]
    \textbf{Mass Equation} : \par
    Since geostrophic velocity is divergent free then $O(1)$ mass equations is 
    \[ \frac{\p \tdrho w_0}{\p z} = 0\]
    and $O(\epsilon)$, 
    \begin{equation}
        \nabla \cdot \bu_1 + \frac{1}{\tdrho} \left( \frac{\p \tdrho w_1}{\p z}\right) = 0 \label{first order mass equation}
    \end{equation}
    \textbf{Buoyancy Equation} : \par
    $O(1)$ :
    \[ \Bu w_0 = 0\]
    and $O(\epsilon)$:
    \begin{equation}
        \frac{\D_g b_0}{\D t} + \Bu w_1 = 0 \label{first order buoyancy equation}
    \end{equation}
    Now we take the \textbf{Curl} of Eq \ref{first order momentum equation}, note that 
    \[\nabla \times (\bm{k} \times \bu_1) = \bm{k}\nabla \cdot \bu_1 - \underbrace{u_1 \nabla \cdot \bm{k}}_{=0} + \underbrace{(\bu_1 \cdot \nabla)\bm{k}}_{ =0} - \underbrace{(\bm{k} \cdot \nabla)\bu_1}_{=0} = \bm{k} \nabla \cdot \bu_1\] 
    Define the geostrophic vorticity : 
    \[ \xi_g = \nabla \times \bu_g\]
    Then Eq \ref{first order momentum equation} becomes 
    \begin{align*}
        \frac{\D_g \xi_g}{\D t} + \beta v_0 &= -f_0\nabla \cdot \bu_1 \mn{this equation is already in $\bm{k}$ direction so the unit vector is dropped} \\
        \intertext{Plug in Eq \ref{first order mass equation},}
        &= \frac{f_0}{\tdrho} \frac{\p \tdrho w_1}{\p z}\\
        \intertext{Plug in Eq \ref{first order buoyancy equation} to replace $w_1$} 
        &= -\frac{f_0}{\tdrho} \underbrace{\frac{\p}{\p_z} \left(\tdrho \Bung \frac{\D_g b_0}{\D t}\right)}_{\equiv I}
    \end{align*}
    Now we examine $I$, normally in QG theory, we assume $L_d$ is a constant. Thought from its definition, $N$ cound actually depends on $z$. Since $\nabla \tdrho =0$, we can put the first two terms into the material derivative. \mn{Here we use the fact that $N$ is constant}
    \[ I = \p_z \left(\tdrho \Bung\right) \Dg{b_0} + \tdrho \Bung \p_z \Dg{b_0} \equiv I_1 + I_2\]
    Let's go back to the Hydrostatic balance equation, \mn{we haven't use it yet.}
    For $O(1)$:
    \[ \frac{\p \phi_0}{\p z} = b_0 \quad + \quad f_0 \bm{k} \times \bu_g = -\nabla \phi_0 \qrq \bm{k} \times \frac{\p \bu_g}{\p z} = -\frac{\nabla b_0}{f_0}\]
    Then 
    \[ I_2  =\tdrho \Bung \p_z \Dg{b_0}= \tdrho \Bung \left[\Dg{\p_z b_0} + \underbrace{\p_z \bu_g \cdot \nabla b_0}_{ = 0}\right]\]
    Therefore
    \[ I = \p_z \left(\tdrho \Bung\right) \Dg{b_0} +  \tdrho \Bung \Dg{\p_z b_0} = \frac{\D_g}{\D t} \left[\frac{\p}{\p_z}\left(\tdrho \Bung b_0\right)\right]\]
    Then evantually we have 
    \begin{equation}
        \frac{\D_g}{\D t} \left[ \xi_g + f + \frac{f_0}{\tdrho}\frac{\p}{\p_z}\left(\tdrho \Bung b_0\right)\right] = 0 \label{temp QGPV}
    \end{equation}
    We can rewrite this equation using Streamfunction in a more simple form. Recall Eq \ref{Buoyancy equation}, we have 
    \[ b_0 = \frac{\p \phi_0}{\p z}\]
    From Eq \ref{Geostrophic balance equation}, the Kinematic Pressure can be expressed in terms of geostrophic streamfunction 
    \[ u_g = -\p_y \psi_g \quad v_g = \p_x \psi_g \qquad \text{where} \quad \boxed{\phi_0 = f_0 \psi_g} \qrq \xi_g = \nabla^2 \psi_g\]
    Then Eq \ref{temp QGPV} becomes 
    \begin{equation}
         \frac{\D_g}{\D t} \left[ \nabla^2 \psi_g + f + \frac{f_0^2}{\tdrho}\frac{\p}{\p_z}\left(\tdrho \Bung \frac{\p \psi_g}{\p z}\right)\right] = 0 \label{dimensionless QGPV}
    \end{equation}
    Restore the dimensions
    \begin{equation}
        \frac{\D g}{\D t} \left[ \nabla^2 \psi_g + f + \frac{f_0^2}{\tilde{\rho}}\frac{\p }{\p_z} \left( \frac{\tdrho}{N^2} \frac{\p \psi}{\p z}\right)\right] = 0 \label{QGPV equation}
    \end{equation}

    \subsection{Ertel PV Conservation}

    \begin{theorem}
        The Ertel PV, denoted usually as $q$ or $Q$, is defined as:
        \begin{equation}
        Q = \frac{\boldsymbol{\omega}_a \cdot \nabla \psi}{\rho} \label{Ertel PV}
        \end{equation}Where:
        \begin{enumerate}
            \item $\boldsymbol{\omega}_a = \nabla \times \mathbf{u} + 2\boldsymbol{\Omega}$ is the absolute vorticity.
            \item $\psi$ is a conserved scalar (like potential temperature $\theta$ or density).
            \item $\rho$ is the density.
        \end{enumerate}
    \end{theorem}
    The derivation starts from the Momentum Equation
    \[\frac{\D \mathbf{u}}{\D t} + \frac{1}{\rho}\nabla p = -2\boldsymbol{\Omega} \times \mathbf{u}\]
    Use the vector identity then take Curl
    \[ (\mathbf{u} \cdot \nabla)\mathbf{u} = \nabla(\frac{1}{2}\mathbf{u}^2) - \mathbf{u} \times (\nabla \times \mathbf{u})\]
    We have 
    \[\frac{\partial \boldsymbol{\omega}_a}{\partial t} - \nabla \times (\mathbf{u} \times \boldsymbol{\omega}_a) = \nabla \times \left( -\frac{1}{\rho}\nabla p \right)\]
    Then apply the vector identity
    \[\nabla \times (\frac{1}{\rho}\nabla p) = \frac{1}{\rho^2} \nabla \rho \times \nabla p\]
    We get 
    \begin{equation}
        \frac{\D \boldsymbol{\omega}_a}{\D t} = (\boldsymbol{\omega}_a \cdot \nabla)\mathbf{u} - \boldsymbol{\omega}_a (\nabla \cdot \mathbf{u}) + \frac{\nabla \rho \times \nabla p}{\rho^2}
    \end{equation}

    \section{Surface Quasi-Geostrophic Equations}
    The surface Quasi-Geostrophic Equation takes the problem to the next step, how could we retrive the interior motion from surface measurements such as SSH and SST. Recall the Buoyancy Equation
    \begin{equation}
        \frac{\D b_1}{\D t} + wN^2 = 0
    \end{equation}
    At the surface, $z = \eta$, the boundary condition yeilds that $w = 0$. We denote the surface buoyancy as $b_s$ and surface velocity $\bu_s$ Then 
    \[ \frac{\p b_s}{\p t} + \bu_s \cdot \nabla b_s = 0\]
    and 
    \[ b_s = f_0 \frac{\p \psi}{\p z}\Big|_{z = 0}\]
    \begin{explanation}
    The critical principle of SQG is to view surface buoyancy as a PV sheet. Since 
    \[ \int_{0}^{\epsilon} f + \nabla^2 \psi_g +  \frac{f_0^2}{\tilde{\rho}}\frac{\p }{\p_z} \left( \frac{\tdrho}{N^2} \frac{\p \psi}{\p z}\right) \mathrm{d} z = 0\]
    Then we impose a boundary condition
    \[ \frac{\p \psi}{\p z} \Big|_{z = \epsilon} = \frac{b_s}{f_0} \qrq \int_{0}^{\epsilon} \frac{b_s}{f_0} = \frac{\p \psi}{\p z}\Big|^\epsilon_0\]
    Compare the latter witht the integral, by defining 
    \[ q_{\text{SQG}} = \nabla^2\psi_g + f + \frac{f_0^2}{\tilde{\rho}}\frac{\p }{\p_z} \left( \frac{\tdrho}{N^2} \frac{\p \psi}{\p z}\right)  + \frac{b_s}{f_0} \delta(z) \]
    We have 
    \[ \frac{\D q}{\D t} = 0 \qquad \frac{\p \psi}{\p z} = 0\]
    Then the surface buoyancy appears in the QGPV equation naturally, it is as if adding an additional PV sheet at the surface. This inspires us to seperate the surface induced dynamics and interior dynamics. 
    \end{explanation}
    \textbf{Interior Dynamics}
    \[\begin{cases}
        q &= \nabla^2 \psi + f + \dfrac{f_0^2}{\tilde{\rho}}\dfrac{\p }{\p_z} \left( \dfrac{\tdrho}{N^2} \dfrac{\p \psi}{\p z}\right) \\
        f_0 \dfrac{\p \psi}{\p z}\Big|_{z = 0} &= 0 \\
        \dfrac{\D q}{\D t} &= 0
    \end{cases}\]
    \subsection{Surface Buoyancy Induced Dynamics}
    \textbf{Surface Dynamics}, this is the Surface Quasi-Geostrophic Dynamics: 
    \[ \begin{cases}
        q &= \nabla^2 \psi + f + \dfrac{f_0^2}{\tilde{\rho}}\dfrac{\p }{\p_z} \left( \dfrac{\tdrho}{N^2} \dfrac{\p \psi}{\p z}\right) = 0 \mn{no interior Potential vorticity} \\
        f_0 \dfrac{\p \psi}{\p z}\Big|_{z = 0} &= b_s \\
        \dfrac{\D b_s}{\D t} &= 0
    \end{cases}\]
    The key assumption for SQG theories is that all the Potential vorticity is injected into the system by surface buoyancy. \mn{surface buoyancy is actually first order, so it is also called surface buoyancy anomoly in some context. }
    
    \subsection{Retriving Vertical Velocity}

    


    \newpage
    \section{\texorpdfstring{$\QGone$}{QG1} Model}
    \label{sec:QGone model}


    The derivation starts from the same set of equations above. We set
    \[ \epsilon = \Ro = \frac{U}{fL} \qquad \text{Bu} = \left( \frac{NH}{fL}\right)^2\]
    The Derivation starts from Eq \ref{dimensionless buoyancy equation}, \ref{dimensionless contiuity equation for ocean}, \ref{dimensionless hydrostatic balance equation} and \ref{dimensionless momentum equation}. The \hyperref[Ertel PV]{Ertel Potential Vorticity} is conserved. In this case, the conserved quantity is the \textbf{total Buoyancy} since Eq \ref{dimensionless buoyancy equation} we define it as 
    \[ b_{\text{tot}} = N^2z + b\]
    and
    \[ \bm{\omega}_a = \nabla_3 \times (u,v,0) + f\hat{z} = (-v_z, u_z, f + \xi)\]
    Then 
    \begin{equation} Q = \underbrace{fN^2}_{\text{Background}} + \underbrace{(N^2\xi + fb_z)}_{\text{Linear terms, QGPV}} + \underbrace{(\xi b_z - v_zb_x + u_zb_y)}_{\text{Nonlinear terms}}\end{equation}
    and
    \[ q_{\text{QG}} = N^2 \xi + fb_z\]
    Use the vector identity 
    \[ \nabla_3 \cdot (\bm{\omega} b) = \bm{\omega} 、\cdot \nabla_3 b + \underbrace{b (\nabla \cdot \bm{\omega})}_{=0, \bm{\omega} = \nabla \times \cdot}\]
    We note that 
    \[ \xi b_z - v_z bx + u_z b_y = \bm{\omega} \cdot \nabla_3 b\]
    where 
    \[ \bm{\omega} = (-v_z, u_z, \xi) = (-v_z, u_z, v_x - u_y)\]
    Then after scaling anlysis we have 
    \[Q = N^2 f + \epsilon q\]
    where 
    \begin{equation}
        q = N^2 \xi + \frac{1}{\Bur} fb_z + \frac{\epsilon}{\Bur} \nabla_3 \cdot (\bm{\omega} b) \label{Dimensionless PV}
    \end{equation}
    The first two term is the calssical Quasi Geostrophic Potential Vorticity. 
    \begin{explanation}
        Since here $N$ is constant and we assume the Boussinesq Equation, the third term in Eq \ref{temp QGPV} becomes,
        \[ \frac{f_0}{\cancel{\rho}} \cancel{\rho} \frac{\p b}{\p z} 
        \left( \frac{fL}{NH}\right)^2 \sim fb_z \frac{1}{\Bur}\]
        The only difference here is the third second order ageostrophic quadratic correction we normally ignore in classical QG theories. 
    \end{explanation}

    \subsection{\texorpdfstring{$\QGone$}{QG1} Vector Field}
    In order to facilitate asymptotic approximation, the velocity field $\bu$ are writen as the Curl of a vector field $\bm{A}$. \mn{Changing three variables in velocity to $\bm{A}$}. One convention is writing 
    \begin{equation}
        \bm{A} = (-G, F, \Phi)
    \end{equation}
    and 
    \[ \bu = \nabla \times \bm{A}\]
    \[
    \begin{aligned}
    u & =-\Phi_y-F_z \\
    v & =\Phi_x-G_z \\
    w & =F_x+G_y
    \end{aligned}
    \]
    The Horizontal Vorticity in vector potential form is 
    \[ \xi = v_x - u_y = \Phi_{xx} - G_{zx} + \Phi_{yy} + F_{zy} = \nabla^2 \Phi + F_{zy} - G_{zx}\]
    However, this $\bm{A}$ isn't uniquely correspond to a velocity field. Since Gradient is Curl free, $\bu$ admits a gauge freedom in that the transformation 
    \[ \bm{A} \rightarrow \bm{A} + \nabla_3 \Gamma\]
    Left $\bu$ unchange. Now instead of assuming $\bm{A}$ is divergent free. \mn{if $\bm{A}$ is divergent free, then this is adding an additioal information to the system and $\bm{A}$ can be uniquely determined.} We fix the buoyancy to be roughly scaled version of $\nabla_3 \cdot \bm{A}$. 
    \[ \nabla_3 \cdot \bm{A} = -G_x + F_y - \Phi_z \qrq b = f\Phi_z + \Bur \frac{N^2}{f}(G_x - F_y)\]
    The advantage can be seen by calculating the QGPV Eq \ref{Dimensionless PV} (\textbf{Dimensionless form}) : 
    \begin{equation}
        q_{\text{QG}} = N^2 \xi + \frac{f}{\Bur}b_z = N^2(\Phi_{xx} + \Phi_{yy}) + \frac{f^2}{\Bur}\Phi_{zz} = \mcalL \Phi
    \end{equation}
    Where 
    \begin{equation}
        \mcalL = N^2 \nabla^2 + \frac{f^2}{\Bur}\p_{zz} \label{modified laplacian}
    \end{equation} 
    Then $\Phi$ is the only component related to the QGPV compare to dependence on all three component of velocity in the classical QGPV. When is the Buoyancy unchanged? \par
    \begin{explanation}
        Suppose we add a gradient to the original Vector Potential 
        \[ \bm{A} \rightarrow \bm{A} + \nabla_3 \Gamma\]
        Then 
        \[ G \rightarrow G - \Gamma_x  \quad F \rightarrow F + \Gamma_y \rightarrow \Phi \rightarrow \Phi - \Gamma_z\]
        And 
        \[ b_{new} = \underbrace{\left[ f\Phi_z + \{Bu\}\frac{N^2}{f}(G_x - F_y) \right]}_{b_{origin}} - \underbrace{\left[ f\Gamma_{zz} + \{Bu\}\frac{N^2}{f}(\Gamma_{xx} + \Gamma_{yy}) \right]}_{b_{change}}\]
        So in order to have $b_{new} = b_{origin}$. We have 
        \begin{equation}
            b_{change} =  f\Gamma_{zz} + \{Bu\}\frac{N^2}{f}(\Gamma_{xx} + \Gamma_{yy})  = \boxed{\mcalL(\Gamma) = 0} \label{unchanged buoyancy}
        \end{equation}
    \end{explanation}
    Some Discussions
    \begin{enumerate}
        \item\textbf{In a triply-periodic domain} : \par Eq \ref{unchanged buoyancy} implies that $\Gamma$ is constant in the domain.
        \item \textbf{Rigid Lid and flat bottom} : \par since $w = 0$ at upper and lower boundary, we have 
        \[ F_x + G_y = 0\]
    \end{enumerate}

    \subsection{Evolution and Inversion Equations of \texorpdfstring{$\QGone$}{QG1}}
    Assume 
    \[ G^0 = F^0 = 0\]
    and 
    \[ \epsilon \ll 1 \qquad \Bur \sim O(1)\]
    Then expand $\bm{A}$ asymptotically we have 
    \begin{subequations}    
        \begin{align}
            u &= -\Phi_y^0 - \epsilon(\Phi_y^1 + F_z^1) \\
            v &= \Phi_x^0 + \epsilon(\Phi_x^1 - G_z^1) \\
            w &= 0 + \epsilon(F_x^1 + G_y^1) \\
            b &= f\Phi_z^0 + \epsilon f\left( \Phi_z^1 + \Bur \frac{N^2}{f^2}(G_x^1 - F_y^1)\right) \label{buoyancy in QG+1}
        \end{align}
    \end{subequations}
    Evaluate Eq \ref{Dimensionless PV} we have up to $O(\epsilon)$. 
    \begin{align}
        q &= N^2 (\nabla^2\Phi^0 + \nabla^2\Phi^1 - G_{zx}^1 + F_{zy}^1) + \frac{f^2}{\Bur}\Phi_z^0 + \epsilon \frac{f^2}{\Bur}\left( \Phi_z^1 + \Bur \frac{N^2}{f^2}(G_x^1 - F_y^1)\right) \nonumber \\
        &+ \underbrace{\left[-\Phi_{xz}^0, -\Phi_{yz}^0,\nabla^2\Phi^0\right]}_{\text{This is } \bm{\omega} = (-v_z, u_z, \xi)} \cdot \underbrace{\left[f\Phi_{zx}^0, f\Phi_{zy}^0, f\Phi_{zz}^0\right]}_{\text{this is } \nabla_3 b} \nonumber \\
        &= \boxed{\mcalL(\Phi^0) + \epsilon \mcalL(\Phi^1) + \epsilon \frac{f}{\Bur}\Big(-|\nabla\Phi_z^0|^2 + \Phi_{zz}^0\nabla^2\Phi^0\Big) + O(\epsilon^2)} \label{potential vorticity up to first order}
    \end{align}
    The surface Buoyancy from 
    \begin{subequations}
        \begin{align}
            b^t &= f\Phi_z^0\Big|_{z = 0} + f\Phi_z^1\Big|_{z = 0} + O(\epsilon^2) \\
            b^b &= f\Phi_z^0\Big|_{z = -H} + f\Phi_z^1\Big|_{z = -H} + O(\epsilon^2)  
        \end{align}
    \end{subequations}

    To complete the inversion, we need to relate ageostrophic vertical streamfunctions $F^1$ and $G^1$ to $\Phi^0$. We start with the first order primitive equations 
    \begin{subequations}
        \begin{align}
                \frac{\D u}{\D t} - f v^1 &= -p_x^1 \\
                \frac{\D v}{\D t} + fu^1 &= -p_y^1 \label{first order primitive b}\\
                \frac{\D b}{\D t} + \Bur N^2  w^1 &= 0 \label{first order primitive c}\\
                p_z^1 &= b^1
        \end{align}
    \end{subequations}
    Take the difference of $z$-derivative of $f$ times Eq \ref{first order primitive b} and $x$-derivetive Eq \ref{first order primitive c} we derive 
    \begin{equation}
        \mcalL(F^1) = \frac{2f}{\Bur} J(\Phi_z^0, \Phi_x^0) \label{F^1 and Phi^0 relation}
    \end{equation}
    and similarly 
    \begin{equation}
        \mcalL(G^1) = \frac{2f}{\Bur} J(\Phi_z^0, \Phi_y^0) \label{G^1 and Phi^0 relation}
    \end{equation}



    \section{\texorpdfstring{$\SQGone$}{SQG1} Model}

    From the Geostrophic Balance Equation we have 
    \[ \nabla^2 p^0 = f\xi^0\]
    To obtain the next order balance equation, we use Eq \ref{buoyancy in QG+1}, \ref{F^1 and Phi^0 relation} and \ref{G^1 and Phi^0 relation}.
    \begin{equation}
        \begin{aligned}
            \nabla^2 b^1 - f\xi_z^1 &= f\left( \nabla^2 \Phi_z + \text{Bu}\frac{N^2}{f^2}\nabla^2(G_x - F_y)\right) - f\nabla^2\Phi_z - f\partial_{zz}(G_x - F_y) \\
            &= \frac{\text{Bu}}{f}\mathcal{L}(G_x - F_y) \\[1.5ex]
            &\quad \text{(Here apply Eq \ref{F^1 and Phi^0 relation} and Eq \ref{G^1 and Phi^0 relation})}  \\[1.5ex]
            &= 2\partial_z J(\Phi_x^0, \Phi_y^0)
        \end{aligned}
    \end{equation}
    Recall that $p_z = b$ by definition then we have 
    \begin{equation}
        \boxed{\nabla^2 p^1 - f\xi^1 = 2J(\Phi_x^0, \Phi_y^0)}
    \end{equation}
    This model captures the ageostrophic at first order and the term on the right hand side represents the \textbf{cyclogeostrophic correction}. In SQG, interior potential vorticity is $0$. Then from Eq \ref{potential vorticity up to first order}, since $q^1 = 0$. Then 
    \[
        \mcalL(\Phi^1) = \frac{f}{\Bur} \left( |\nabla\Phi_z^0|^2 - \Phi_{zz}^0 \nabla^2 \Phi^0 \right)
    \]
    Using $\mcalL(\Phi^0)= 0$. We have 
    \begin{equation}
        \mcalL(\Phi^1) = \frac{f}{N^2 \Bur} \left(N^2 |\nabla \Phi^0_z|^2 + \frac{f^2}{\Bur}\Phi_{zz}^0\Phi_{zz}^0\right) \label{main equation for Phi^1}
    \end{equation}
    Together with Eq \ref{F^1 and Phi^0 relation}, Eq \ref{G^1 and Phi^0 relation} and boundary conditions. 
    \[ \Phi_z^1 = F^1 = G^1 = 0 \qquad \text{at } z = 0\]
    Further we assume all potential vanish at infinity. 
    \begin{explanation}
        For this system, the goal is to know $\Phi^1, G^1$ and $F^1$. We recall that first order potential $G^0 = F^0 = 0$. And $\Phi^0$ can be obtain from geostrophic balance. This is a system of decoupled linear elliptic (Poisson) equations. \begin{enumerate}
            \item  Decoupled: You can solve for $\Phi^1$, $F^1$, and $G^1$ independently of each other.
            \item Linear/Poisson: Each equation takes the form $\mathcal{L}(\text{Potential}) = \text{Source Term}$.
            \item Dependence: The "Source Terms" on the right-hand side (RHS) are known quantities derived entirely from the zeroth-order solution $\Phi^0$ (which is obtained from the observed SSH/buoyancy)
        \end{enumerate} 
    \end{explanation}
    The inversion is carried on by decomposing potential into interior and surface part. The interior part satisfies the main equations but ignore the boundary condition. The surface part solves the homogeneous problem and but corrects the boundary conditions. For $\Phi^1$, we first notice that 
    \[\Phi_{int}^1 = \frac{f}{2N^2\Bur}\Phi_z^0\Phi_z^0\]
    Is a solution to Eq \ref{main equation for Phi^1}. Then it remains to solve the surface part given by 
    \begin{align}
        \mcalL(\Phi_{sur}^1) &= 0 \\
        \Phi_{sur,z}^{1,t} &= C_b - \partial_z \Phi_{int,z}^{1,t} = C_b - \frac{f}{\Bur N^2}\Phi_z^{0,t}\Phi_{zz}^{0,t}
    \end{align}

\end{document}