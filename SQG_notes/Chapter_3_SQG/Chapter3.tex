\documentclass[../SQG_notes.tex]{subfiles}

\begin{document}

\section{Simulation Set Up}

In this section, I briefly describe the simulation set up from the most trivial case all the way up to the true problem. We start with a brief introduction about how the inversion process proceed. 

\subsection{Inversion Process}
From hydrostatic balance, the surface pressure is propotional to the surface sea height (SSH). 
\begin{equation}
    \eta = \frac{p\{z = 0\}}{g} \label{surface pressure}
\end{equation}
In all the QG models, the pressure is expanded asymptotically in the form of 
\begin{equation}
    p_{total} \sim p^0 + \epsilon p^1
\end{equation}
The first order pressure is directly related to the first order potential $\Phi^0$ by the formula 
\[ p^0 = f\Phi^0\]
As from geostrophic balance we have 
\[ p^0_y = -f u^0 = -f \p_y \Phi^0 \qquad p^0_x = f v^0 = f \p_x \Phi^0\]
Now we have 
\[ \eta^0 = \frac{f\Phi^0}{g}\]
Similarly from the first order correction we have 
\[ \eta^1 = \frac{p^1}{g}\]
So in general, considering the first order correction to the hydrostatic balance Eq \ref{surface pressure}, we have 
\begin{equation}
    \eta \sim \eta^0 + \epsilon\eta^1 = \frac{f\Phi^0}{g} +\epsilon \frac{p^1}{g}
\end{equation}
This is the equation 40 in Ryan's note. \par

Now remember our goal is to use $\eta$ to invert for $\Phi^0$. What is the relationship between $p^1$ and $\Phi^0$? The relationship is given by eq \ref{cyclogeostrophic correction}. We copy it here 
\[
\boxed{\nabla^2 p^1 - f\zeta^1 = 2J(\Phi_x^0, \Phi_y^0)}
\]
Where $J$ is the Jacobian operator. This is a non-linear relationship, recall that $\zeta^1$ is related to the potentials via Eq \ref{vorticity in terms of potentials}. So 
\[
\nabla^2 p^1 = f\left(\nabla^2\Phi^1 + F_{zy}^1 - G_{zx}^1\right) + 2J(\Phi_x^0, \Phi_y^0)
\]
Here the relationship between $\Phi^1, F^1$ and $G^1$ are given by Eq \ref{Fourier transform of Phi^1}, \ref{F^1 fourier transform} and \ref{G^1 fourier transform}. Then the whole inversion problem is clear.

\subsection{Pre-Defined Velocity Fields}
We start with a very simple case to play around with this model. \par

\subsubsection{Toy Model Setup}
The toy model start with a pre-defined potential $\Phi^0$. 

\end{document}