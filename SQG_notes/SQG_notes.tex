\documentclass{article}

\usepackage{D:/Documents/Latex/Latex_Template/NotesTeXV3_test}
\usepackage{subfiles}
\usepackage{natbib}
\usepackage{booktabs} % For professional table rules (lines)
\usepackage{ltablex}

\special{dvipdfmx:config z 0}

\input{D:/Documents/Latex/Latex_Template/Custom_Commands.tex}
\newcommand{\QGone}{\text{QG}^{+1}}

\begin{document}
    \title{{Lecture Notes on Surface Quasi-Geostrophic}\\{\normalsize{\itshape Academic Semester: Fall 2025}}}
	\author{Thomas Li}
	\affiliation{
	Undergrauate Student at New York University Shanghai\\
	}
	\emailAdd{jl15535@nyu.edu}
	\maketitle
    \pagestyle{fancynotes}
	\newpage

\begin{tabularx}{\textwidth}{@{} c X c X @{}} 

% --- Header ---
\caption{Glossary of Variables and Operators}\\
\toprule
\multicolumn{2}{c}{\textbf{Variables and Operators}} & \multicolumn{2}{c}{\textbf{}} \\
\cmidrule(r){1-2} \cmidrule(l){3-4}
\textbf{Symbol} & \textbf{Variable and Operators} & \textbf{Symbol} & \textbf{Description} \\
\midrule
\endfirsthead

% --- Header for Subsequent Pages (if it spans multiple pages) ---
\toprule
\multicolumn{2}{c}{\textbf{Group 1 (Cont.)}} & \multicolumn{2}{c}{\textbf{Group 2 (Cont.)}} \\
\cmidrule(r){1-2} \cmidrule(l){3-4}
\textbf{Symbol} & \textbf{Description} & \textbf{Symbol} & \textbf{Description} \\
\midrule
\endhead

% --- Footer ---
\bottomrule
\multicolumn{4}{r}{\textit{Continued on next page...}} \\
\endfoot
\bottomrule
\endlastfoot

% --- CONTENT ROWS ---
% Use & to separate columns and \\ to end the row.

$\mathbf{v}(x,y,z,t) = (u,v,w)$          & Full 3-dimensional Velocity & $\nabla$ & 2D Gradient Operator 
               \\ 
\addlinespace[10pt] % Adds vertical space between rows

$\mathbf{u}(x,y,z,t) = (u,v)$& 2-dimensional velocity, in $x$ and $y$ direction& $\nabla_3$  
&  3D Gradient  \\ 
\addlinespace[10pt]

 $\widehat{\cdot}$& Dimensionless Variable & $\dfrac{\D}{\D t}$& Material Derivative \\
\addlinespace[10pt]

$\psi$& Geostrophic Streamfunction defined in Eq& $\nabla^2$& 2D laplacian operator (zonal and meridional) \\
\addlinespace[10pt]

$(\bm{i}, \bm{j}, \bm{k})$& Unit vector in zonal, meridional and vertical direction& $\widehat{\cdot}$& Wide hat for Fourier Transform \\
\addlinespace[10pt]

&&& \\
\addlinespace[10pt]

&&& \\
\addlinespace[10pt]

% ... Add more rows here ...

\end{tabularx}

\section{Question for Meetings}

\subsection{0108}

\begin{enumerate}
	\item So the $\QGone$ model is still geostrphic? How does it improve the accuracy of submesoscale dynamics prediction who is not geostrophic?
	\item What is the Physical Meaning of the $\mcalL$ operator acting on $\Gamma$ which is the Gauge transform is 0. 
	\item Leading order of the Potential Terms are about Geostrophiic Balanace?
	\item Sign Difference?
	\item Briefly walk me through how is the code constructed? What are some of the potential \textbf{difficulties}? How do we overcome this. 
\end{enumerate}


\newpage
\subfile{Literature_review/Chapter_0.tex}

\newpage
\subfile{Chapter1_SQG/Chapter1.tex}

\newpage
\subfile{Chapter_2_SQG1/Chapter2.tex}
\vfill
\bibliographystyle{abbrvnat}
\bibliography{citation}

\end{document}
